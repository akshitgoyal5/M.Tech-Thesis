\chapter{Approximate Algorithm}
\label{sec:approx_algo}
Exact \textit{replica} construction (Section \ref{subsubsec:replica-gen}) for a
subgraph pattern requires searching for every instance of the pattern in the
data graph using the \textit{replica} of its parent pattern. In
\textsc{FindAllInstances} (Algorithm \ref{algo:complete-instances}), we identify
all instances one-by-one in a depth-first guided search. However, performing
this computation can be expensive since subgraph isomorphism is known to be an
\textit{NP-hard} problem and thus the computational complexity is worst-case
exponential in the pattern size. Moreover, the number of instances of a pattern
generally grows exponentially with increasing density and size of the data
graph. As a result, a \emph{complete} enumeration of instances such as in
Algorithm \ref{algo:complete-instances} does not scale well to dense or large
graphs. This establishes the need for a faster strategy for \textit{replica}
construction.

We now make an important observation: to match a vertex $v\in V(G)$ and
appropriate edges incident to $v$ in $V(replica(R))$  and $E(replica(R))$
respectively for a pattern $R$, the identification of \textit{all} instances of
$R$ containing $v$ might not be necessary. In other words, by identifying more
instances containing $v$ than might be necessary, we do not gain any additional
information about the \textit{replica} graph structure. This suggests that
repeated traversals on vertices during the course of identifying instances can
perhaps be reduced without losing structural information about the
\textit{replica}.

Algorithm \ref{algo:approx-instances} describes an alternative strategy to
Algorithm \ref{algo:complete-instances} to implement the idea for accelerated
\textit{replica} construction by reducing repeated traversals on
"already-explored" vertices. We attempt to avoid enumerating those instances
during construction that do not provide new information about the \emph{replica}
structure as a strategy to reduce computational complexity. Before discussing
this method, we define some additional data structures for the \textit{replica}:
% However, the strategy also could miss identifying a few relevant instances in
% certain cases which could possibly result in the loss of some information
% about the \textit{replica} graph and its mappings lists (Section
% \ref{subsubsec:replica-ds}). Consequently, the \textit{replica} $R_a$ thus
% obtained would be an \textbf{approximation} of the complete \emph{replica}
% $R_c$ such that $V_{R_a} \subseteq V_{R_c}$ and $E_{R_a} \subseteq E_{R_c}$ .
% In Section \ref{sec:experiments}, we demonstrate empirically that such an
% approximation scheme, while enabling computational tractability, does not
% significantly affect quality of the top-\textit{k} results. 

%!!!write that misses are insignificant as proved in the experiments write that
%!!!approximation means missing correct not including incorrect mappings

% Algorithm \ref{algo:approx-instances} describes this approximation scheme.


\para{$\bullet$ Potential children set ${P_{v,c}}$} \\
(defined for each $v\in V(replica(Q))$ mapped to $p\in V(Q)$ for each child
$c\in children(p,Q) \in V(Q)$): stores every replica vertex $w\in
V(replica(Q))$ such that $c\in children(p,Q)$ can potentially map to $w$ (i.e. $w$ is
a candidate for $c$).

\para{$\bullet$ Enumerated set ${E_v}$} \\
(defined for each $v\in V(Q)$): stores every vertex $w\in replica(Q)$ that maps
to $v$ such that: $\forall c\in children(v,Q),\ P_{v,c}=\emptyset$. In other
words, $E_{v}$ stores all its mappings that have been "fully-explored" for all
child mappings. 

% $v$ has been
% fully-explored for all possible child mappings below it.

\para{$\bullet$ Confirmed children set ${C_{v,c}}$} \\
(defined for each $v\in V(replica(Q))$ mapped to $p\in V(Q)$ for each child
$c\in children(p,Q) \in V(Q)$): stores every replica vertex $w\in V(replica(Q))$
such that: (1) there exists at least one instance including the mappings $(p, v)$ and
$(w,c)$, and (2) $w \in E_{c}$  

%(Algorithms \ref{algo:search} and \ref{algo:complete-instances}) requires and
% correlation computation 

Algorithm \ref{algo:approx-instances} is a recursive algorithm. In the general
(recursive) case (lines 5-30), it begins by invoking \textsc{NextQueryEdge}
which returns one edge at a time from $E(Q)$ in the order of the rooted $DFS\
List$. Edge $e(p,c)$ thus returned connects vertices $p,c \in V(Q)$ such that
$c\in children(p)$ as per the $DFS\ List$ and $c$ is the pattern vertex to be
matched next ($p$ is matched to $v \in V(replica(Q))$). The algorithm then
checks for the existence of the potential children set for storing candidate
vertices for matching $c$, given that $p$ is mapped to $v$. If this set
$P_{v,c}(\subseteq V(replica(Q)))$ does not exist, it is computed by invoking
\textsc{FilterCandidates} which stores all vertices $w\in V(replica(Q))$ such
that: (1) $w$ is contained in the \textit{replica} adjacency list of $v$; (2)
$c$ exists in the inverse mapping list of $w$. That is, \forall $w \in P_{v,c}$,
$c\in InverseMap(w)$; (3) The edge label of $(v,w)\in E(replica(Q))$ matches
that of $(p,c)\in E(Q)$. Thus, set $P_{v,c}$ is constructed once when a match
for $c$ is being sought for the first time with $p$ having been mapped to $v$,
and indexed. Similarly, confirmed children set $C_{v,c}$ is initialized as an
empty set and indexed. In the event of any subsequent visit, the algorithm
simply considers candidates remaining in $P_{v,c}$ for matching $c$. Next, for
every vertex $w \in P_{v,c}$ such that $w$ has not already been mapped to some
other pattern vertex in the current $instance$, the algorithm attempts the
mapping $(c,w)$ in $instance$ and recursively calls
\textsc{FindAllInstancesApprox} to match remaining pattern vertices following
the $DFS\ List$ of edges. It sets a boolean $InstanceFound$ to $True$ if it
finds at least one $instance$ in the recursive call and appends the set of all
found instances to $\mathbb{I}$. During the recursive call, if $w$ gets inserted
into the enumerated set for $c$ ($E_c$), it is transferred from $P_{v,c}$ to
$C_{v,c}$. Finally, $(c,w)$ is removed from $instance$ and the next valid
candidate in $P_{v,c}$ tried as a possible matching for $c$ in the following
iteration. Suppose, the algorithm fails to find any $instance$ even after
considering every $w\in P_{v,c}$. (A possible scenario where this could happen
is if $P_{v,c}=\emptyset$, after all original candidates have been confirmed and
transferred to $C_{v,c}$). In such a scenario, $InstanceFound$ would remain
false after the first loop. The algorithm would then start iterating over
$C_{v,c}$, the set of confirmed children till the time an $instance$ is found
for some $w' \in C_{v,u}$ and $InstanceFound$ is set to $True$. Finally, if the candidates set $P_{v,c'}$ for \textbf{every} $c'\in children(p)$
is exhausted, $v$ is appended to the enumerated set for $p (E_{p})$. This means that
\textit{all} possible candidate child mappings have been traversed and
confirmed. Note that in lines ??-??, child mapping $w$ if inserted into $E_{c}$
gets tranferred from $P_{v,c}$ to $C_{v,c}$. This is because any subsequent searches for possible mappings of $c$ at $v$
can ignore all $w' \in C_{v,c}$ since such a $w'$ has been completely
"enumerated" and thus matching it to $c$ would result in repeated traversals
over already explored portions of the graph. The \textit{Base Case} (lines 1-3) occurs when the algorithm finds an instance
of the child pattern $R$ (i.e., $|instance|=|V(R)|$). The algorithm
records the mappings of every $l'\in Leaves(Q)$ in the corresponding enumerated
sets, since there cannot exist any further child mappings below leaves. Finally,
the $instance$ is returned.


\begin{algorithm}
	\caption{\textsc{FindAllInstancesApprox}}\label{algo:approx-instances}
	\dontprintsemicolon
	% \begin{algorithmic}[1] \REQUIRE data graph: $G$, parent pattern: $Q$,
	% replica of $Q$: $replica(Q)$, child pattern: $R$, parent edge: $(u, v)\in
	% E(G)$ mapped to $(u', v')\in E(R)$, $DFS\ List$ of $Q$ rooted at
	% $extending\
	% index$, $instance$, $\mathbb{I}$ \ENSURE all $instances\ \mathbb{I}$ of
	% pattern $R$ in $G$ such that $\forall I \in \mathbb{I},\ (u, v)\in I$ as a
	% mapping of $(u',v')$\\ 
	\KwIn{Graph $G$, parent: $Q$, $replica(Q)$, child: $R$, parent edge: $(u,
			v)\in E(G)$ mapped to $(u', v')\in E(R)$, $DFS\ List$ of $Q$ rooted
			at $extending\ index$, $instance$, $\mathbb{I}$} \KwOut{relevant
			$instances\ \mathbb{I}$ of pattern $R$ in $G$ such that $\forall I
			\in
			\mathbb{I},\ (u, v)\in E(I)$ as a mapping of $(u',v')$}

	\nonl $Base\ Case:$\;
	\uIf {$|instance| = |V(R)|$}
	{
		
		Update $E_{l'}$ for each  $l'\in Leaves(Q)$\;
		\nonl $[[E_{l'}\leftarrow E_{l'} \cup \{instance(l')\}, \forall l' \in Leaves(Q)]]$ \;
		\Return{$instance$}\;
	} 
		$Recursive\ Case:$\; 
		$e(p,c)\coloneq$ \textsc{NextQueryEdge($DFS\ List$, ...)}\; \nonl $[[\ (p,c) \in E(Q),\ c
		\in children(p)]]$\; $v\coloneq$ Mapping of $p$ in $instance$\; \nonl
		$[[\ v\in V(replica(Q))\wedge (p,v)\in instance]]$\;
	\uIf{${P_{v,c}}$ \textup{does not exist}}
	{${P_{v,c}}\coloneq$ \textsc{FilterCandidates($instance$, $v$, $c$, $...$)}
		\; \nonl $[[\forall w \in P_{v,c}\ ((w\in V(replica(Q))\wedge (w\in
		adjList(v)) \wedge (c\in InverseMap(w))\wedge (L(v,w)=L(p,c)) )]]$\;
		${C_{v,c}}\coloneq \emptyset$\;}
	% \uIf{$v \in V(replica(Q))$ \textup{is mapped to} $v'$ \textup{for the
	% first time}}
	% {
	%   $c':=$ \textsc{NextQueryVertex($DFS\ List$, $v'$)} \; ${P_{v,c'}}:=$
	%   \textsc{FilterCandidates($instance$, $v$, $c'$, $...$)} \;
	%   ${C_{v,c'}}:=\emptyset$\; %     \FOR{\textbf{each} adjacent edge
	%   $e(v,w)$ of $v$ in $replica(Q)$} \IF{$e$ %     maps to the corresponding
	%   $child\ edge$ in $DFS\ List$} \STATE $\mathbb{P} %     \leftarrow
	%   \mathbb{P} \cup w$ \ENDIF \ENDFOR % \ENDIF
	% }
	$boolean$ $InstanceFound \coloneq$ {\sf False}\; \ForEach {$w \in P_{v,c}$
		\textup{such that $w$ is not yet matched}} {$instance\leftarrow
		instance\cup \{(c,w)\}$\; $\mathbb{I'} \leftarrow $
		\textsc{FindAllInstancesApprox($R$, $e$, $instance$, $DFS\ List$)} \;
		\uIf {$\mathbb{I'} \neq \emptyset$}
		{$\mathbb{I}\leftarrow \mathbb{I}\ \cup\ \mathbb{I'}$\;
			$InstanceFound\leftarrow$ {\sf True} \;}
		\uIf {$w \in E_{c}$}
		{	$P_{v,c}\leftarrow P_{v,c} \setminus \{w\}$\; $C_{v,c}\leftarrow
				C_{v,c} \cup \{w\}$\;} $instance\leftarrow instance \setminus
				\{(c,w)\}$\;}
	% \While{$InstanceFound$ \textup{is} {\sf False}}
	% {
	\ForEach{$w' \in C_{v,c}$ \textup{such that $w'$ is not yet matched and}
		$InstanceFound$ \textup{is} {\sf False}} {$instance\leftarrow
		instance\cup \{(c,w')\}$\; $\mathbb{I'} \leftarrow $
		\textsc{FindAllInstancesApprox($R$, $e$, $instance$, $DFS\ List$)} \;
		\uIf {$\mathbb{I'} \neq \emptyset$}
		{$\mathbb{I}\leftarrow \mathbb{I}\ \cup\ \mathbb{I'}$\;
			$InstanceFound\leftarrow$ {\sf True} \;} $instance\leftarrow
			instance \setminus \{(c,w')\}$\;
		% \IF{edge $e'(v,w')$ is \textbf{not} already in $instance$} \STATE
		% $instance\leftarrow instance\cup e'$ \STATE $\mathbb{I'}\leftarrow$
		% {\sf find\ all\ instances($R$, $e'$, $instance$, $DFS\ List$)}
		% \IF{$\mathbb{I'}$ is \textbf{not} empty} \STATE $\mathbb{I}\leftarrow
		% \mathbb{I}\ \cup\ \mathbb{I'}$ \STATE $Found\leftarrow True$ \STATE
		% \textbf{break} \ENDIF \ENDIF Recursively enumerate all instances
		% $(lines[12:16])$ \;
	}
	% }
	\uIf{$\ \forall c' \in children(p),\ P_{v,c} = \emptyset$}
	{ $E_{p} \leftarrow E_{p} \cup v$} \Return{$\mathbb{I}}\;
\end{algorithm}
