\chapter{Conclusions}
\label{sec:conclusions}
In this thesis, we presented a novel approach to compute \textit{correlations}
between two subgraphs in a single large graph, where we defined
\textit{correlations} on the basis of a distance metric. We introduced a novel
method to group instances and defined the \textit{correlation value} $(\tau)$ as
the count of such \textit{instance groups} for a pattern in proximity to those
of another pattern. For computing all instances, we proposed a
\textsc{TreeSearch} backtracking algorithm. The algorithm constructs what we
call the \textit{Replica} structure - an occurrence graph of a pattern in a data
graph. However in a dense or large graph, computing the \textit{replica} is
computationally intensive since subgraph isomorphism is an \textit{NP-hard}
problem and the number of instances of a pattern generally increases
exponentially with increasing graph density. To solve this problem, we proposed
an Approximation scheme that constructs the \textit{Replica} approximately. We
observed good performance of the Approximation algorithm on large datasets, as
reported by the \textit{Kendall-Tau's coefficient}, \textit{Jaccard coefficient}
and \textit{Percentage error in $\tau$ values} - which were close to perfect
values for almost all datasets we tested on. 